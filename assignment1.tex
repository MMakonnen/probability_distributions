% arara: pdflatex: { shell: yes }
% arara: pythontex: {verbose: yes, rerun: modified }
% arara: pdflatex: { shell: yes }

\documentclass[assignments]{subfiles}

\begin{document}


\section{Assignment 1}
\label{sec:org9cbca82}


\subsection{Have you read well?}
\label{sec:have-you-read}


\begin{exercise}
In your own words, explain what is
\begin{enumerate}
\item a joint PMF, PDF, CDF;
\item a conditional PMF, PDF, CDF;
\item a marginal PMF, PDF, CDF.
\end{enumerate}
\end{exercise}


\begin{exercise}
We have two r.v.s $X$ and $Y$ with the joint PDF $f_{X,Y}(x,y) = 2 \1{x\leq y}$? 
\begin{enumerate}
\item Are $X$ and $Y$ independent? 
\item Compute $F_{X,Y}(x,y)$. 
\end{enumerate}
\end{exercise}

\begin{exercise}
Correct (that is, is the following clam correct?)? We have two continuous r.v.s $X, Y$. Even though the joint CDF factors into the product of the marginal it is still possible in general that the joint PDF does not factors into a product of marginals PDFs of $X$ and $Y$.
\end{exercise}

\begin{exercise}
Express Bayes' formula for two rvs $X$ and $Y$ in terms of the joint CDF, i.e., provide a formula.
\end{exercise}

\begin{exercise}
What is a contingency table?
\end{exercise}

\begin{exercise}
Let $X$ be uniformly distributed on the set $\{0,1,2\}$ and let $Y \sim Bern(1/4)$; $X$ and $Y$ are independent.
Present a contingency table for the $X$ and $Y$.
What is the interpretation of the column sums the table?
What is the interpretation of the row sums of the table? Suppose you change some of the entries in the table, are $X$ and $Y$ still independent?
\end{exercise}

\begin{exercise}
Apply the chicken-egg story.
A machine makes items on a day.
Some items, independent of the other items, are failed (i.e., do not meet the quality requirements).
What is $N$, what is $p$, what are the `eggs' in this context, and what is the meaning of `hatching'?
What type of `hatching' do we have here?
\end{exercise}

\begin{exercise}
Apply the chicken-egg story. Families enter a zoo in a given hour. Some families have one child, other two, and so on. 
What are the `eggs' in this context, and what is the meaning of `hatching'? 
\end{exercise}


\begin{exercise}
Claim: We have two rvs $X$ and $Y$ on $\R^{+}$. It is given that $F_{X,Y}(x,y) = F_X(x)F_Y(y)$ for $x,y \leq 1/3$. Then  $X$ and $Y$ are necessarily independent. 
\end{exercise}

\begin{exercise}
`The man on the street' says that two throws of a die are independent, but does not mention the joint CDF.
What do you think of this claim of independence?
(Hint, from this exercise you should memorize this: \textbf{independence is a property of the joint CDF, not of the rvs}.)
\end{exercise}


\begin{exercise}
I select a random guy from the street, his height $X\sim\Norm{1.8, 0.1}$, and I select a random woman from the street, her height is $Y\sim\Norm{1.7, 0.08}$.
I claim that since I selected the man and the woman independently, their heights are independent.
Briefly comment on this claim.
\end{exercise}


\begin{exercise}
Correct? For any two rvs $X$ and $Y$ on $\R^{+}$ with marginals $F_{X}$ and $F_{Y}$. Then $\P{X\leq x, Y\leq y} = F_{X}(x) F_{Y}(x)$.
\end{exercise}

\begin{exercise}
Theorem 7.1.11. What is the meaning of the notation $X|N=n$?
\end{exercise}

\begin{exercise}
Correct? $X, Y$ two discrete rvs with CDF $F_{X,Y}$. We can compute the PDF as $\partial_{x}\partial_{y} F_{X,Y}(x,y)$.
\end{exercise}


\subsection{Exercises at about exam level}
\label{sec:below-exam-level}



\begin{exercise}\label{ex:1}
We throw an unbiased die with six sides; the result of the $i$th throw is $X_i$. 
\begin{enumerate}
\item What is the sample space of the two throws $(X_{1}, X_2)$?
\item What is the joint CDF?
\item What is the joint PMF?
\item  Marginalize out $X_2$ to show that $\P{X_1=5} = 1/6$.
\item Use the fundamental bridge and indicators to compute $\P{X_1>X_2}$.
\item Use the fundamental bridge and indicators to compute $\P{|X_1-X_2| < 1 } = 1/6$.
\item Use the fundamental bridge and indicators to compute $\P{|X_1-X_2|\leq 1 }$.
\item How would you use simulation to estimate  $\P{|X_1-X_2|\leq 1 }$?
\end{enumerate}
\end{exercise}

\begin{exercise}
We select a random married couple (a man and a woman). His height is $X\sim \Norm{1.8, 0.1}$, her height is $Y\sim \Norm{1.7, 0.08}$ in meters. 
\begin{enumerate}
\item What is the sample space of $(X, Y)$?
\item If your answer to question 1 is correct, you must have noticed that potentially the height of the man and the woman can be negative. Is this a problem for this model? 
\item What is the joint CDF?
\item What is the joint PDF?
\item  Marginalize out $Y$ to show that $X\sim \Norm{1.8, 0.1}$.
\item Use the fundamental bridge and indicators to write  $\P{X>Y}$ as an integral. You don't have to solve the integral.
\item Use the fundamental bridge and indicators to write  $\P{|X-Y| < 0.1 }$ as an integral. You don't have to solve the integral.
\end{enumerate}
\end{exercise}

\begin{exercise}
This is about the simplest model for an insurance company that I can think of.
We start with an initial capital $I_0=2$.
The company receives claims and contributions every period, a week say.
In the $i$th period, we receive a contribution $X_{i}$ uniform on the set $\{1, 2,\ldots,10\}$ and a claim $C_i$ uniform on $\{0, 1, \ldots 8\}$.
\begin{enumerate}
\item What is the interpretation of $\bar I_{n} = \min{I_{i} : 0\leq i\leq n}$? 
\item What is the meaning of $I_1=I_0+X_1-C_1$?
\item What is the meaning of $I_2=I_1+X_2-C_2$?
\item What is the interpretation of $I_1'=\max{I_0-C_1,0} + X_1$?
\item What is the interpretation of $I_2'=\max{I_1'-C_2,0} + X_2$?
\item What is  $\P{I_1 < 0}$?
\item What is  $\P{I_1' < 0}$?
\item What is  $\P{I_2 < 0}$?
\item What is  $\P{I_2' < 0}$?
\item Provide an interpretation in terms of the inventory of rice, say, at a supermarket for $I_{1}$ and $I_{1}'$.
\item Provide also an interpretation in terms of a degradation and repair process of an item.
  (if you find this difficult, search a bit on the web on reliability theory.)
  Comment on how good you think this model is to analyze such degradation and repair processes.
\end{enumerate}
\end{exercise}

\begin{exercise}
We have a machine that consists of two components. The machine works as long as not both components have failed. Let $X_i$ be the lifetime of component $i$. 
\begin{enumerate}
\item What is the interpretation of $\min{X_1, X_{2}}$?
\item What is the interpretation of $\max{X_1, X_{2}}$?
\item If $X_1, X_2$ iid $\sim \Exp{10}$ (in hours), what is the expected time until the machine fails? 
\item If $X_1, X_2$ iid $\sim \Exp{10}$ (in hours), what is the probability that the machine is still `up' (i.e., not failed) at time $T=50$? 
\end{enumerate}
\end{exercise}

\begin{exercise}
Let $X$ be the result of the throw of a coin.
It is given that $\P{X=H} = p = 1-\P{X=T}$.
When $X=H$, we choose a fair die with 4 sides with values $1,2,3,4$, when $X=T$ we choose a fair die with 6 sides with values $1,\ldots,6$.
Let $Y_i$ be the value of the $i$th throw with the die.
\begin{enumerate}
\item What is the PMF of $X$ and $Y_1$?
\item Marginalize  the answer of part a to show that $\P{X=H} = p$. 
\item What is $\P{Y_{1}=1}$?
\item What is $\P{X=H \given Y_1=1}$?
\item What is $\P{X=H \given Y_1=1, Y_2=2}$?
\item What is $\P{X=H \given Y_1=Y_2=\cdots=Y_n = 1}$?
\end{enumerate}
\end{exercise}



\begin{exercise} %wk
Assume that $X$ has the Cauchy distribution. 
\begin{enumerate}
\item Does $\E{\frac{X}{X^2+1}}$ exist? If so, find its value. 
\item Does $\E{\frac{|X|}{X^2+1}}$ exist? If so, find its value. 
\end{enumerate}
\end{exercise}


\subsection{Coding skills}
\label{sec:coding-skills}



\begin{exercise} 
Use simulation to estimate the answer of BH.7.1. Run the code below and explain line 9 of python code or line of the R code. 


Then run the code for a larger sample, e.g, \verb|num=1000| or so, but remove the prints of \verb|a|, \verb|b|, and \verb|succes|, because that will fill your screen with numbers you don't need.
Only for small simulations such output is handy so that you can check the code.

Compare the value of the simulation to the exact value.


\begin{minted}[]{python}
import numpy as np

np.random.seed(3)

num = 10

a = np.random.uniform(size=num)
b = np.random.uniform(size=num)
success = np.abs(a - b) < 0.25
print(a)
print(b)
print(success)
print(success.mean(), success.var())
\end{minted}


\begin{minted}[]{R}
a <- 3
\end{minted}

Challenge (not obligatory): If you like, you can include a plot of the region (in time) in which Alice and Bob meet, and put marks on the points of the simulation that were `successful'. 
\end{exercise}



\begin{exercise}
Let $X\sim\Exp{3}$.
Find a simple expression for $\P{1 < X \leq 4}$ and compute the value.
Then use simulation to check this value.
Finally, use numerical integration to compute this value. What are the numbers? Explain lines lines 11, 21 and 26 of the python code. 

\begin{minted}[]{python}
import numpy as np
from scipy.stats import expon
from scipy.integrate import quad

labda = 3

X = expon(scale=labda).rvs(1000)
# print(X)
print(X.mean())

success = (X > 1) * (X < 4)
# print(success)
print(success.mean(), success.std())


def F(x):  # CDF
    return 1 - np.exp(-labda * x)


def f(x):  # density
    return labda * np.exp(-labda * x)


print(F(4) - F(1))

I = quad(f, 1, 4)
print(I)
\end{minted}
\end{exercise}


\begin{exercise}\label{ex:2}
How many ping pong balls fit into an Airbus Beluga?
One way to answer this is as follows.
According to this \href{https://en.wikipedia.org/wiki/Airbus\_Beluga}{wikipage} the cargo volume $V$ of this airplane is $1500 \m^{3}$.
But this number is based on the physical dimensions that is available to store containers, tanks, and so on.
So, I estimate the volume as about twice that amount, i.e., $V = 2500 \m^{3}$. 
The volume of a ping pong ball is $v = 4 \pi r^3/3  = \py{4*3.14*8/3} \cm^{3}$ with $r=2$ cm.
A plain division gives \py{2500/33.5} ping pong balls.
Note, I left out the $10^{6}$ conversion from meters to cm, and I do not take into the sphere packing factor.
Besides that, I hope you agree with me that providing an result with the precision as given here is plain ridiculous.
(But from reason incomprehensible to me, even professional econometricians like to report results with 10 digits or more, without questioning the precision.)


However, I know that the volumes of an air plane and a ping pong ball is an estimate, rather than a precise number as assumed above.
It seems to be better to approximate $V$ and $v$ as rvs.
Let's assume that
   \begin{align*}
V & \sim N(2500, 500), & v  & \sim N(33.5, 0.5),
\end{align*}
where the variances express my trust in my guess work.
What is now the mean of $N = V/v$ and its std?
In fact, finding the closed form expression for the distribution of $N$ is not entirely simple.
However, with simulation it's easy to get an estimate.

Use the code below to provide these estimates, and explain line 11 of the python code.


Contrary to BH.7.1.25  if you run the code below, you'll see that $\E N < \infty$, and, in fact, very near to the deterministic answer. But isn't this strange?  We divide two normal random variables, 
just like BH.7.1.25, but there the expectation is infinite.  Comment on the difference.

The numerical results suggest the interesting guess $\V N \approx \V V * \V v$, but  is this true more generally? In~\cref{ex:3} we research this problem in more detail.

\begin{minted}[]{python}
import numpy as np
from scipy.stats import norm

num = 500

np.random.seed(3)

V = norm(2500, 500)
v = norm(33.5, 0.5)

N = V.rvs(num) / v.rvs(num)
print(N.mean(), N.std())

print(2500/33.5)
print(np.sqrt(500*0.5))
\end{minted}
\end{exercise}

\subsection{Challenges}
\label{sec:above-exam-level}

\begin{exercise} %wk
Consider the again the chicken-egg story (BH 7.1.9): A chicken lays a random number of eggs $N$ an each egg independently hatches with probability $p$ and fails to hatch with probability $q = 1-p$. Let $X$ be the number of eggs that hatch and let $Y$ be the number of eggs that do not hatch, so $X+Y = N$. 
For $N \sim \Pois{\lambda}$ it is shown in BH 7.1.9 that $X$ and $Y$ are independent. This exercise asks for the converse. Assume that $X$ and $Y$ are independent. Prove that there exists a $\lambda > 0$ such that $N \sim \Pois{\lambda}$.
\end{exercise}


\begin{exercise}\label{ex:3}
This is a continuation of~\cref{ex:2}.  TBD.
 
\begin{verbatim}
Als V[V] = 0 of V[v] = 0, is rechts 0, links niet perse.
Maar er is nog een betere reden waarom dit niet kan kloppen. Definieer V' = 2V, v'=2v en N' = V'/v'. Dan N=N', dus links blijft gelijk. Maar rechts wordt 16 keer zo groot...

Ik denk dat ik het heb opgelost. Merk op dat we aannemen dat V en v onafhankelijk zijn. Dus
V[V/v] = E[V^2/v^2] - E[V/v]^2 = E[V^2]E[1/v^2] - E[V]^2E[1/v]^2 = V[V]V[1/v] + E[V]^2 E[1/v^2]  + E[V^2] E[1/v]^2 - 2 E[V]^2E[1/v]^2= V[V]V[1/v] + E[V]^2 V[1/v] + V[V] E[1/v]^2.
Als we er van uit gaan dat we V en v voldoende precies weten, geldt er V[V]V[1/v] << E[V]^2 V[1/v] + V[V] E[1/v]^2.
De eerste-orde Taylorbenadering van 1/v rond E[v] is 2/E[v] - v/E[v]^2. We kunnen dus benaderen V[1/v] \approx  V[v]/E[v]^4.
Dezelfde benadering geeft ook E[1/v]^2 \approx 1/ E[v]^2. Dus het resultaat is al met al  V[V/v]\approx  E[V]^2 V[v]/E[v]^4 + V[V] / E[v]^2 = E[V/v]^2( V[v]/E[v]^2 +  V[V]/E[V]^2 ).
Ofwel: V[V/v]/E[V/v]^2=  V[v]/E[v]^2 +  V[V]/E[V]^2, dus SCV(V/v) \approx SCV(V)  + SCV(v).
Overigens: de benadering zit hier in SCV(1/v) \approx SCV(v). Immers, SCV(V/v) = SCV(V)  + SCV(1/v) wegens onafhankelijkheid. 

\end{verbatim}

\end{exercise}

\end{document}

\subsection{TODO for another year}

Here are some ideas for another time. You can neglect this. 

\begin{exercise}
For another time: afrondregels, bijv \(2.11*3= 6\), omdat \(3\) maar op 1 digit bekend is. Wat is de kwaliteit van deze afrondregel?
More generally,  check density for sum,product, and ratio of two normal (or other distributions)  by means of simulations.
\end{exercise}

