% arara: pdflatex: { shell: yes }
% arara: pythontex: {verbose: yes, rerun: modified }
% arara: pdflatex: { shell: yes }


\documentclass[a4paper]{article}
\newcommand{\wk}[1]{\textbf{#1}} 


\usepackage[check]{optional}
\usepackage{preamble}


\author{Nicky D. van Foreest and student assistents}
\date{\today}
\title{Probability distributions EBP038A05: 2020-2021\\
Assignments}
\begin{document}
 

\maketitle


\section*{General information}
\label{sec:orgb865fed}

Here we just provide the exercises of the assignments.  For information with respect to grading we refer to the  course manual.


The assignments contain several sections.
The first section is meant to help you read the book well and become familiar with definitions and concepts of probability theory.
These questions are mostly simple checks, not at exam level, but lower.
The second section contains some exercises at about the exam level to get you started.
Most of the selected exercises of the book are also at about (or just a bit above) exam level.
The third section is about coding skills.
We explain the rationale presently.
The final section with challenges is for those students that like a challenge; the problems are above exam level.


You have to get used to programming and checking your work with computers, for instance by using simulation.
The coding exercises address this skill.
You should know that much of programming is `monkey see, monkey do'.
This means that you take code of others, try to understand it, and then adapt it to your needs.
For this reason we include the code to answer the question.
The idea is that you copy the code, you  run it and include the numerical results in your report. You should be able to explain how the code works. For this reason we include questions in which you have explain how the most salient parts of the code works. 

We include python and R code, and leave the choice to you what to use.
In the exam we will also include both languages in the same problem, so  you can stay in the language you like.
You should know, however, that many of you  will need both languages (and perhaps yet others, such as java, or SQL, or \ldots) later in life. 

The rules: 
\begin{enumerate}
\item For each assigment you have to turn in a pdf document typeset in \LaTeX{}. Include a title, group number, student names and ids, and date.
\item We expect brief answers, just a sentence or so, or a number plus some short explanation. The idea of the assignment is to help you studying, not to turn you in a writer. 
\item When you have to turn in a graph, provide decent labels and a legend, ensure the axes have labels too.
\end{enumerate}



\subfile{assignment1}
\subfile{assignment2}
\subfile{assignment3}
\subfile{assignment4}
\subfile{assignment5}
\subfile{assignment6}
\end{document}






\section{Lots of  ideas, very messy currently}
\label{sec:lots-ideas-very}



\subsection{Exercise rock, scissors, paper with multiple players.}
\label{sec:org351adb7}




\section{A/B testing}
\label{sec:org180b96a}
\subsubsection{Introduction}
\label{sec:org41a02a9}

In an A/B test the problem is to find out which of two alternatives is the better, in that it yields higher profit (webpages), cures more patients (medicine), and so on. In this assignment you have to study different policies to find out the better of two alternatives. 

A/B testing is a very interesting topic, and lies at the heart of \emph{reinforcement learning}, see wikipedia for background, in particular with respect to machine learning, automatic car driving, and so on. 


\subsubsection{{\bfseries\sffamily TODO} Use beta priors}
\label{sec:org9297a86}

\subsubsection{{\bfseries\sffamily TODO} Interesting to link this to the game of Alice and Bob in Van der Plas}
\label{sec:orgafbfbd0}
\url{http://jakevdp.github.io/blog/2014/06/06/frequentism-and-bayesianism-2-when-results-differ/}

\subsection{Estimating the number of tanks}
\label{sec:org0700c8f}


\subsection{Koffiedingen van Douwe Egberts}
\label{sec:orgbe91c90}

\subsection{Batchgewijs testen door samen te voegen}
\label{sec:org7f21ad2}



\subsection{Sport uitslagen}
\label{sec:org9a46c5b}

Som 7.48 gaat over hoeveel maxima optreden. Wat is de verdeling van de maxima? Wat is de toename? Relatie met extreme waarde verdelingen. 

\subsection{Random variable code en die van Niels}
\label{sec:org57ced54}



\subsubsection{code}
\label{sec:orga01964c}


\begin{minted}[]{python}
import numpy as np

np.random.seed(3)

sample_no = 300
rvs_no = 2

#a = np.random.uniform(size=[sample_no, rvs_no])
labda = 2
a = np.random.exponential(scale=labda, size=[sample_no, rvs_no])
a.sort(axis=1)
#a
a.mean(axis=0), a.std(axis=0)
\end{minted}

\begin{minted}[]{python}
labda/2, 3*labda/2
\end{minted}



\section{Bootstrapping example}
\label{sec:org6006a3a}


\end{document}
