\documentclass[a4paper]{article}
\newcommand{\wk}[1]{\textbf{#1}} 
%\usepackage[all-solutions-at-end]{optional}
\usepackage[check]{optional}


\usepackage{preamble}


\author{Nicky D. van Foreest}
\date{2020:11:15}
\title{Probability distributions: Assignments\\\medskip
\large EBP038A05}
\begin{document}

\maketitle


\section*{General information}
\label{sec:orgb865fed}

Here we just provide the exercises of the assignments.  For information with respect to grading we refer to the  course manual.

\begin{enumerate}
\item For each assigment you have to turn in a pdf document typeset in \LaTeX{}. Include a title, group number, student names and ids, and date.
\item When you have to turn in a graph, provide decent labels and a legend, ensure the axes have labels too.
\item Whenever you have to program or simulate something, include your code. 
\end{enumerate}

\subfile{assignment1}

\end{document}






\section{Lots of  ideas, very messy currently}
\label{sec:lots-ideas-very}



\subsection{Exercise rock, scissors, paper with multiple players.}
\label{sec:org351adb7}




\section{A/B testing}
\label{sec:org180b96a}
\subsubsection{Introduction}
\label{sec:org41a02a9}

In an A/B test the problem is to find out which of two alternatives is the better, in that it yields higher profit (webpages), cures more patients (medicine), and so on. In this assignment you have to study different policies to find out the better of two alternatives. 

A/B testing is a very interesting topic, and lies at the heart of \emph{reinforcement learning}, see wikipedia for background, in particular with respect to machine learning, automatic car driving, and so on. 


\subsubsection{{\bfseries\sffamily TODO} Use beta priors}
\label{sec:org9297a86}

\subsubsection{{\bfseries\sffamily TODO} Interesting to link this to the game of Alice and Bob in Van der Plas}
\label{sec:orgafbfbd0}
\url{http://jakevdp.github.io/blog/2014/06/06/frequentism-and-bayesianism-2-when-results-differ/}

\subsection{Estimating the number of tanks}
\label{sec:org0700c8f}


\subsection{Koffiedingen van Douwe Egberts}
\label{sec:orgbe91c90}

\subsection{Batchgewijs testen door samen te voegen}
\label{sec:org7f21ad2}



\subsection{Sport uitslagen}
\label{sec:org9a46c5b}

Som 7.48 gaat over hoeveel maxima optreden. Wat is de verdeling van de maxima? Wat is de toename? Relatie met extreme waarde verdelingen. 

\subsection{Random variable code en die van Niels}
\label{sec:org57ced54}

\section{Maximum of independent r.v.s; Example 5.6.5}
\label{sec:orgb2a351d}

\begin{enumerate}
\item Make 1d array of uniform 0,1 random rvs
\item compute mean and variance
\item Why include a seed
\item include seed
\item Make [n, p] matrix, n samples along rows
\item Sort along axis  1
\item Sort along axis 0
\item Compute mean and std along axis 0
\item Make large number of data, with e.g sample-no = 1000, and redo the above
\item Change to exponential distribution
\item Show how to use the online documentation for np.random.exponential
\item Show the effect of the scale parameter
\item Compare mean to theoretical value
\item Change to geometric distribution
\end{enumerate}


\subsubsection{code}
\label{sec:orga01964c}


\begin{minted}[]{python}
import numpy as np

np.random.seed(3)

sample_no = 300
rvs_no = 2

#a = np.random.uniform(size=[sample_no, rvs_no])
labda = 2
a = np.random.exponential(scale=labda, size=[sample_no, rvs_no])
a.sort(axis=1)
#a
a.mean(axis=0), a.std(axis=0)
\end{minted}

\begin{minted}[]{python}
labda/2, 3*labda/2
\end{minted}



\section{Bootstrapping example}
\label{sec:org6006a3a}


\end{document}
