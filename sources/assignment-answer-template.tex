\documentclass[a4paper,11pt]{article}

\usepackage[english]{babel}
\usepackage{mathtools,amsthm,amssymb,amsmath}
\usepackage{a4wide}

\usepackage{fouriernc}
\usepackage{hyperref}
\usepackage[capitalize]{cleveref}


\usepackage{minted}
\setminted[python]{linenos=true}
\setminted[python]{frame=lines}
\setminted[R]{linenos=true}
\setminted[R]{frame=lines}

\newcommand{\N}{\mathbb{N}}
\newcommand{\R}{\mathbb{R}}
\newcommand{\Z}{\mathbb{Z}}
\DeclareMathOperator*{\argmin}{arg\,min}

\newcommand{\abs}[1]{\left\vert#1\right\vert}
\newcommand{\given}{\,\middle|\,}
\newcommand{\Bern}[1]{\mathrm{Bern}(#1)}
\newcommand{\Bin}[1]{\mathrm{Bin}(#1)}
\newcommand{\Exp}[1]{\mathrm{Exp}(#1)}
\newcommand{\FS}[1]{\mathrm{FS}(#1)}
\newcommand{\Geo}[1]{\mathrm{Geo}(#1)}
\newcommand{\Norm}[1]{\mathrm{Norm}(#1)}
\newcommand{\Pois}[1]{\mathrm{Pois}(#1)}
\newcommand{\Unif}[1]{\mathrm{Unif}(#1)}
\renewcommand{\P}[1]{\,\mathsf{P}\left\{#1\right\}}
\newcommand{\E}[1]{\,\mathsf{E}\left[#1\right]}
\newcommand{\EE}[2]{\,\mathsf{E}_{#1}\left[#2\right]}
\newcommand{\V}[1]{\,\mathsf{V}\left[#1\right]}
\newcommand{\cov}[1]{\,\mathsf{Cov}\left[#1\right]}
\renewcommand{\d}[1]{\,\textrm{d}#1}
\newcommand{\1}[1]{\,I_{#1}} % indicator

\author{Claire Gun (s123456) and Clark Rifle (s654321)}
\date{\today}
\title{Prob dist., assignment 1\\
  2020-2021
  }
\begin{document}

\maketitle

Just to help you, read also the \verb|newcommands| above. Using such commands can make your life considerably easier.

\section{Exercise/question/problem 1}

In our understading, taking the product of two numbers comes down to repeating taking the sum of couple of times.

\section{Problem 2}

When we run the simulation, see the code and parameters below, we get the answer 50

\begin{minted}[]{python}
a = 10
b = 5
print(a*b)
\end{minted}


\begin{minted}[]{R}
a <- 5
b <- 10
b*a
\end{minted}

\section{Exercise/question/problem 3}

The graph is like this (check on \texttt{overleaf} (see the web), how to include graphs in \LaTeX).

\end{document}
