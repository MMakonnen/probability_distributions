\documentclass[assignments]{subfiles}

\begin{document}
\section{Assignment 2}


\subsection{Have you read well?}


\begin{exercise}
Example 7.2.2. Write down the integral to compute $\E{(X-Y)^{2}}$. You don't have to solve the integral.
\end{exercise}

\begin{exercise}
Give a brief example of a situation where it might be more convenient to employ the correlation instead of the covariance and explain why.
\end{exercise}

\begin{exercise}
In queueing theory  the concept of squared coefficient of variance $SCV$ of a rv $X$ is very important. It is defined as $C = \V{X}/(\E X)^{2}$. Is the SCV of $X$ equal to $\text{Corr}(X,X)$? Can it happen that $C>1$?
\end{exercise}


\begin{exercise}
Using the definition of Covariance (Definition 7.3.1) derive the expression $\cov{X,Y}=\E{XY}-\E{X}\E{Y}$. Use this finding to show why independence of X and Y implies their uncorrelatedness (note that the converse does not hold).
\end{exercise}

\begin{exercise}
Let X,Y be two r.v.s and let a,b be two arbitrary scalars. Express $\cov{a(U+V), b(U-V)}$ in terms of  $\V{U}$, $\V{U}$ and $\cov{U,V}$ (by using the expression obtained in the previous question).
\end{exercise}

\begin{exercise}
Prove the key properties of covariance 1 to 5 on page 327 of the book (page 338 pdf).
\end{exercise}

\begin{exercise}
Come up with a short illustrative example in which the random vector $\mathbf{X} = (X_1, \ldots, X_6)$ follows a Multinomial Distribution with parameters  $n=10$ and $\mathbf{p}=(\frac{1}{6}, ..., \frac{1}{6}) \in \R^{6}$.
\end{exercise}

\begin{exercise}
Is the following claim correct? If the r.v.s $X, Y$ are both normally distributed, then $(X, Y)$ follows a Bivariate Normal distribution.
\end{exercise}

\begin{exercise}
Let $(X,Y)$ follow a Bivariate Normal distribution, with $X$ and $Y$ marginally following $\mathcal{N}(\mu,\sigma^2)$ and with correlation $\rho$ between $X$ and $Y$.
\begin{enumerate}
\item Use the definition of a Multivariate Normal Distribution to show that $(X+Y, X-Y)$ is also Bivariate Normal.
\item Find the marginal distributions of $X+Y$ and $X-Y$.
\item Compute $\cov{X,Y}$. Then, write down the expression for the joint PDF of $(X+Y, X-Y)$.
\end{enumerate}
\end{exercise}

\begin{exercise}
Let $X, Y, Z$ be i.i.d. $\mathcal{N}(0,1).$ Determine whether or not the random vector
\begin{align*}
    \mathbf{W} = (X+2Y, 3X+4Z, 5Y+6Z, 2X-4Y+Z, X-9Z, 12X+\sqrt{3}Y -\pi Z)
\end{align*}
is Multivariate Normal. (Explain in words, don't do a lot of tedious math here!)
\end{exercise}


\subsection{Exercises at about exam level}
\label{sec:exercises-at-about}





\begin{exercise}
Take $X\sim\Unif{\{-2, -1, 1, 2\}}$ and $Y= X^2$. What is the correlation coefficient of $X$ and $\eta$?
If we would consider another distribution for $X$, would that change the correlation?
\end{exercise}



\begin{exercise}
We have a machine that consists of two components. The machine works as long as not both components have failed. Let $X_i$ be the lifetime of component $i$.
\begin{enumerate}
\item What is the interpretation of $\min\{X_1, X_{2}\}$?
\item If $X_1, X_2$ iid $\sim \Exp{10}$ (in hours), what is the probability that the machine is still `up' (i.e., not failed) at time $T$?
\item Use the previous result to determine the distribution of $\min \{X_1, X_2\}$.
\item What is the expected time until the machine fails?
\end{enumerate}
\end{exercise}

\begin{exercise}
We have two r.v.s X and Y with the joint PDF $f_{X,Y}(x,y) = \frac{6}{7}(x+y)^2$ for $x, y \in (0,1)$ and 0 else. Also we consider the two r.v.s U and V with the joint PDF $f_{U,V}(u,v) = 2$ for $u, v \in [0,1], u+v \leq 1$ and 0 else.
\begin{enumerate}
\item Compute $\P{X+Y>1}$.
\item Compute $\cov{U,V}$.
\end{enumerate}
(Hint: first draw the area over which you want to integrate, if this does not help check out the discussion board post on exercise 7.13a from the first Tutorial)
\end{exercise}


\subsection{Coding skills}
\label{sec:progr-assignm}


\paragraph{Verify the answers of BH.5.6.5}
Read this example of BH first.
We chop up the exercise in many small exercises..


For python code below, run it for a small number of samples; here I choose \texttt{samples=2}. Read the print statements, and use that to answer the questions below.

\begin{minted}[]{python}
import numpy as np
from scipy.stats import expon

np.random.seed(10)

labda = 4
num = 3
samples = 2

X = expon(scale=labda).rvs((samples, num))
print(X)
T = np.sort(X, axis=1)
print(T)
print(T.mean(axis=0))

expected = np.array([labda / (num - j) for j in range(num)])
print(expected)
print(expected.cumsum())
\end{minted}


\begin{minted}[]{R}
set.seed(10)

labda = 4
num = 3
samples = 2

X = matrix(rexp(samples * num, rate = 1 / labda), nrow = samples, ncol = num)
print(X)
bigT = X
for (i in 1:samples) {
  bigT[i,] = sort(bigT[i,])
}
print(bigT)
print(colMeans(bigT))

expected = rep(0, num)
for (j in 1:num) {
  expected[j] = labda / (num - (j - 1))
}
print(expected)
print(cumsum(expected))
\end{minted}



\begin{exercise}
In line P.11\footnote{Line P.x refers to line x of the Python code.
  Line R.x refers to line x of the R code.}
we print the value of \texttt{X} in line P.10, R.7 and R.8, respectively.
What is the meaning of \texttt{X}?
\begin{solution}
\end{solution}
\end{exercise}

\begin{exercise}
What is the meaning of \texttt{T} in line P.12 (R.11)?
\begin{solution}
\end{solution}
\end{exercise}


\begin{exercise}
What do we print in line P.14, R.14?
\begin{solution}
\end{solution}
\end{exercise}

\begin{exercise}
What is meaning of the variable \texttt{expected}?
\begin{solution}
\end{solution}
\end{exercise}

\begin{exercise}
 What is the \texttt{cumsum} of \texttt{expected}?
\begin{solution}
\end{solution}
\end{exercise}

\begin{exercise}
 Now that you understand what is going on, rerun the simulation for a larger number of samples, e.g., 1000, and discuss the results briefly.
\begin{solution}
\end{solution}
\end{exercise}

\paragraph{On   BH.7.48} Read this exercise first and solve it. Then consider the code below.

\begin{minted}[]{python}
import numpy as np

np.random.seed(3)


def find_number_of_maxima(X):
    num_max = 0
    M = -np.infty
    for x in X:
        if x > M:
            num_max += 1
            M = x
    return num_max


num = 10
X = np.random.uniform(size=num)
print(X)

print(find_number_of_maxima(X))

samples = 100
Y = np.zeros(samples)
for i in range(samples):
    X = np.random.uniform(size=num)
    Y[i] = find_number_of_maxima(X)

print(Y.mean(), Y.var(), Y.std())
\end{minted}


\begin{minted}[]{R}
set.seed(3)

find_number_of_maxima = function(X) {
  num_max = 0
  M = -Inf
  for (x in X) {
    if(x > M) {
      num_max = num_max + 1
      M = x
    }
  }
  return(num_max)
}


num = 10
X = runif(num, min = 0, max = 1)
print(X)

print(find_number_of_maxima(X))

samples = 100
Y = rep(0, samples)
for (i in 1:samples) {
  X = runif(num, min = 0, max = 1)
  Y[i] = find_number_of_maxima(X)
}

print(mean(Y))
print(var(Y))
print(sd(Y))
\end{minted}

\begin{exercise}
Explain how the small function in lines P.6 to P.13 (R.4-R.12) works.
(You should know that \texttt{x += 1} is an extremely useful abbreviation of the code \texttt{x = x + 1}).
\begin{solution}
\end{solution}
\end{exercise}

\begin{exercise}
Explain the code in lines P.25 and P.26 (R.25, R.26).
\begin{solution}
\end{solution}
\end{exercise}

\paragraph{Why is the Exponential Distribution so important?}

At the Paris metro, a train arrives every 3 minutes on a platform.
Suppose that 50 people arrive between the departure of a train and an arrival.
It seems entirely reasonable to me to model the arrival times of the individual people as distributed on the interval \([0,3]\).
What is the distribution of the inter-arrival times of these people?
It turns out to be exponential!
Story BH.13.4.2 provides an explanation (It is not forbidden to read the book beyond what you have to do for this course!); here we use simulation to check this fact.



\begin{minted}[]{python}
import numpy as np

np.random.seed(3)


num = 5 # small sample at first, for checking.
start, end = 0, 3
labda = num / (end - start)  # per minute
print(1 / labda)

A = np.sort(np.random.uniform(start, end, size=num))
print(A)
print(A[1:])
print(A[:-1])
X = A[1:] - A[:-1]
print(X)

print(X.mean(), X.std())
\end{minted}


\begin{minted}[]{R}
set.seed(3)


num = 5
start = 0
end = 3
labda = num / (end - start)
print(1 / labda)

A = sort(runif(num, min = start, max = end))
print(A)
print(A[-1])
print(A[-length(A)])
X = A[-1] - A[-length(A)]
print(X)

print(mean(X))
print(sd(X))
\end{minted}

\begin{exercise}
Explain the result of line P.12 (R.13)
\begin{solution}
\end{solution}
\end{exercise}

\begin{exercise}
Compare the result of  line P.13 and P.14 (R.12, R.13);  explain what is \texttt{A[1:]} (\texttt{A[-1]})
\begin{solution}
\end{solution}
\end{exercise}

\begin{exercise}
Compare the result of  line P.12 and P.14 (R.11 and R.13);  explain what is \texttt{A[:-1]} (\texttt{A[-length(A)]}).
\begin{solution}
\end{solution}
\end{exercise}

\begin{exercise}
 Explain what is \texttt{X} in P.15 (R.14)
\begin{solution}
\end{solution}
\end{exercise}

\begin{exercise}
Why do we compare $1/\lambda$ and \texttt{X.mean()}?
\begin{solution}
\end{solution}
\end{exercise}

\begin{exercise}
Recall that $\E X = \sigma (X)$ when $X\sim \Exp{\lambda}$.
Hence, what do you expect to see for \texttt{X.std()}?
\begin{solution}
\end{solution}
\end{exercise}

\begin{exercise}
 Run the code for a larger sample, e.g. 50, and discuss (very briefly) your results.
\begin{solution}
\end{solution}
\end{exercise}




\subsection{Challenges}
\label{sec:challenges}

This exercise will given an example of how probability theory can pop up in OR problems, in particular in linear programs. It introduces you to the concept of \textit{recourse models}, which you will learn about in the master course Optimization Under Uncertainty. \textbf{Disclaimer: the story is quite lengthy, but the concepts introduced and questions asked are in fact not very hard. We just added the story to make things more intuitive.}

We consider a pastry shop that only sells one product: chocolate muffins. Every morning at 5:00 a.m., the shop owner bakes a stock of fresh muffins, which he sells during the rest of the day. Making one muffin comes at a cost of $c = \$ 1$ per unit. Any leftover muffins must be discarded at the end of the day, so every morning he starts with an empty stock of muffins.

The owner has one question for you: determine the amount $x$ of muffins that he should make in the morning to minimize his production cost. Note that the owner never wants to disappoint any customer, i.e., he requires that $x \geq d$, where $d$ is the daily demand for muffins.

The problem can be formulated as a linear program (LP):
\begin{align}
    \min_{x \geq 0} \{ cx \ : \ x \geq d \}.
\end{align}
For simplicity, we ignore the fact that $x$ should be integer-valued.

\begin{exercise}
Determine the optimal value $x^*$ for $x$ and the corresponding objective value in case $d$ is deterministic.
\end{exercise}

Of course, in practice $d$ is not deterministic. Instead, $d$ is a random variable with some distribution. However, note that the LP above is ill-defined if $d$ is a random variable. We cannot guarantee that $x \geq d$ if we do not know the value of $d$.

You explained the issue to the shop owner and he replies: ``Of course, you're right! You know, whenever I've run out of muffins and a customer asks for one, I make one on the spot. I never disappoint a customer, you know! It does cost me $50 \%$ more money to produce them on the spot, though, you know.''

Mathematically speaking, the shop owner just gave you all the (mathematical) ingredients to build a so-called \textit{recourse model}. We introduce a \textit{recourse variable} $y$ in our model, representing the amount of muffins produced on the spot. Production comes at a unit cost of $q = 1.5 c = \$ 1.5$. Assuming that we know the distribution of $d$, we can then minimize the \textit{expected total cost}:

\begin{align}
    \min_{x \geq 0} \big\{ cx + \E{v(d,x)} \big\},
\end{align}
where $v(d,x)$ is the optimal value of another LP, namely the \textit{recourse problem}:
\begin{align}
    v(d,x) := \min_{y \geq 0} \{ qy \ : \ x + y \geq d \},
\end{align}
for given values of $d$ and $x$. The recourse problem can easily be solved explicitly: we get $y=d-x$ if $d \geq x$ and $y=0$ if $d < x$. So we obtain
\begin{align}
    v(d,x) = q (d - x)^+,
\end{align}
where the operator $(\cdot)^+$ represents the \textit{positive value} operator, defined as
\begin{equation}
    (s)^+ = \begin{cases}
    s &\text{if } s \geq 0,\\
    0 &\text{if } s < 0.
    \end{cases}
\end{equation}

\begin{exercise}
To get some more insight into the model, suppose (for now) that $d \sim U\{10, 20\}$.
    Solve the model, i.e., find the optimal amount $x^*$.
    \textit{Hint: First, compute the value of $\E{v(d,x)}$ as a function of $x$.
      Then find the optimal value of $x$.}
\end{exercise}

\begin{exercise}
 What is the expected recourse cost (expected cost of on-the-spot production) at the optimal solution $x^*$, i.e., compute $\E{v(d,x^*)}$?
\end{exercise}

To solve the model correctly, we need the true distribution of $d$. We learn the following from the shop owner: ``My granddaughter, who's always running around in my shop, is a bit data-crazy, you know, so she's been collecting some data. I remember her saying that `the demand from male and female customers are both approximately normally distributed, with mean values both equal to $10$ and standard deviations of $5$'. She also mentioned something about correlation, but I don't remember exactly, you know. It was either almost $1$ or almost $-1$. I hope this helps!''

Mathematically, we've learned that $d = d_m + d_f$, with $(d_m, d_f) \sim \mathcal{N}(\mu, \Sigma)$, where $\mu = (\mu_m, \mu_f) = (10,10)$ and $\Sigma_{11} = \sigma_m^2 = \Sigma_{22}  = \sigma_f^2 = 5^2 = 25$. Finally, $\Sigma_{12} = \Sigma_{21} = Cov(d_m, d_f) = \rho \sigma_m \sigma_f = 25 \rho$. Also, we know that either $\rho \approx 1$ or $\rho \approx -1$.

\begin{exercise}
 Calculate $x^*$ and the corresponding objective value for the case $\rho = -1$. (Do not read  $\rho=1$, this case is not simple.)
\end{exercise}

\begin{exercise}
Consider the two extreme cases $\rho = 1$ and $\rho = -1$.
In which case will the shop owner have lower expected total costs?
Provide a short, intuitive explanation.
\textit{Hint: you don't have to compute $x^*$ for the case where $\rho = 1$ (this is not easy!)}.
\end{exercise}

\end{document}




\subsection{Measuring inter arrival-times}
\label{sec:orgf820334}

We have a device to measure the time between two arrivals, of jobs for instance, or customers in a shop, or particles in radio-active decay.
After a measurement, the device has to recharge so it cannot measure arrivals that occur within 10 seconds from each other.
Also, if there is no arrival within 50, it resets itself.
Hence, the device cannot easily measure inter-arrival times longer than 1 minute. Assume that the inter-arrival times are exponentially distributed with some unknown \(\lambda\). Let \(\bar x = \sum_{n=1}^N x_{n} /N\) be the sample mean of \(N\) measured inter-arrival times.

\begin{enumerate}
\item Explain that, if \(\lambda \ll 1\) minute,  \(\hat \lambda = \bar x - 10\) is a reasonable estimator for \(\lambda\).
\item How would you obtain a reasonable estimate of \(\lambda\) when \(\lambda\) \(\gg\) 1 minute?
\end{enumerate}



\begin{todo}
Let $X\sim\Exp{\lambda}$ and $Y \sim N(\mu, \sigma)$, independent of $X$.
So, draw $X$ first and let the outcome be $x$; then draw $Y\sim N(x, \sigma)$.
Take $\lambda=4$, $\mu = 5$, $\sigma=3$.
\begin{enumerate}
\item Make a 3D plot of $F_{X,Y}$.
\item Make a 3D plot of $f_{X,Y}$.
\item Plot $f_{X}$, i.e., plot $\mu e^{-\mu t}$.
  Then use simulation to marginalize out $Y$ from $f_{X,Y}$ to obtain $\hat f_X$; we write $\hat f_X$ because it has been obtained from simulation.
  Plot $\hat f_X$ in the same figure as $f_X$, and compare the result.
\item Use simulation to estimate $f_{X|Y}$. Plot this in the same graph for various values of $X=x$.
\item Make a 3D plot of $f$
\end{enumerate}
\end{todo}

\begin{todo}
Let $X\sim\Exp{\lambda}$ and $Y| X \sim N(X, \sigma)$. So,  draw $X$ first and let the outcome be $x$;  then draw $Y\sim N(x, \sigma)$. Take $\lambda = 5$, $\sigma=3$.
\begin{enumerate}
\item Use simulation to estimate $\E Y$.
\item Make a 3D plot of $F_{X,Y}$.
\item Make a 3D plot of $f_{X,Y}$.
\item Plot $f_{X}$, i.e., plot $\lambda e^{-\lambda t}$.
  Then use simulation to marginalize out $Y$ from $f_{X,Y}$ to obtain $\hat f_X$.
  Plot $\hat f_X$ in the same figure as $f_X$, and compare the result.
\item Use simulation to estimate $f_{X|Y}$. Plot this in the same graph for various values of $X=x$.
\item Make a 3D plot of $f$
\end{enumerate}
\end{todo}
