\documentclass[assignments]{subfiles}

\begin{document}


\section{Assignment 2, TBD}


\subsection{Have you read well?}


\begin{exercise}
What is the difference between 1D LOTUS and 2D LOTUS?
\end{exercise}

\begin{exercise}
Example 7.2.2. Write down the integral to compute $\E{(X-Y)^{2}}$. You don't have to solve the integral.
\end{exercise}

\begin{exercise}
In queueing theory  the concept of squared coefficient of variance $SCV$ of a rv $X$ is very important. It is defined as $C = \V{X}/(\E X)^{2}$. Is the SCV of $X$ equal to $\text{Corr}(X,X)$? Can it happen that $C>1$?
\end{exercise}


\subsection{Exercises at about exam level}
\label{sec:exercises-at-about}


\begin{exercise}
Check first BH 7.2.3.
When $X, Y$ iid $\sim \Norm{0, 1}$, then $X-Y\sim\Norm{0,2}$.
However, when $X, Y$ iid $\sim \Pois{\lambda}$, then prove first that $X+Y\sim \Pois{2\lambda}$, but note that $X-Y$ is not $\sim\Pois{0}$.
Explain this difference between the Poisson and normal distribution.
\end{exercise}


\begin{exercise}
Derive the results of BH 7.3.6 without smart tricks. Thus, you have to use the fundamental bridge to show that 
\begin{align*}
\E{M L} &= \E X \E Y = 1, & \E{M} &= 3/2, & \E L &= 1/2, \\
\E{L^{2}} &=  1/4, & \E{M^{2}} &= 2\E{X^{2}} - \E{L^{2}} = 7/2 \\ 
\V{M}&=\E{M^2} - (\E M)^{2}, & \V L &= \E{L^2}-(\E L)^2.
\end{align*}
You can use the document `Memoryless exercursions' to see how to solve these problems. 
\end{exercise}


\begin{exercise}
Take $X\sim\Unif{\{-2, -1, 1, 2\}}$ and $\eta = X^2$. What is the correlation coefficient of $X$ and $\eta$? 
If we would consider another distribution for $X$, would that change the correlation?
\end{exercise}


\begin{exercise}
Let $X$ be the result of the throw of a coin.
It is given that $\P{X=H} = p = 1-\P{X=T}$.
When $X=H$, we choose a fair die with 4 sides with values $1,2,3,4$, when $X=T$ we choose a fair die with 6 sides with values $1,\ldots,6$.
Let $Y_i$ be the value of the $i$th throw with the die.
\begin{enumerate}
\item What is the PMF of $X$ and $Y_1$?
\item Marginalize  the answer of part a to show that $\P{X=H} = p$. 
\item What is $\P{Y_{1}=1}$?
\item What is $\P{X=H \given Y_1=1}$?
\item What is $\P{X=H \given Y_1=1, Y_2=2}$?
\item What is $\P{X=H \given Y_1=Y_2=\cdots=Y_n = 1}$?
\end{enumerate}
\end{exercise}


\begin{exercise}
We have a machine that consists of two components. The machine works as long as not both components have failed. Let $X_i$ be the lifetime of component $i$. 
\begin{enumerate}
\item What is the interpretation of $\min{X_1, X_{2}}$?
\item What is the interpretation of $\max{X_1, X_{2}}$?
\item If $X_1, X_2$ iid $\sim \Exp{10}$ (in hours), what is the expected time until the machine fails? 
\item If $X_1, X_2$ iid $\sim \Exp{10}$ (in hours), what is the probability that the machine is still `up' (i.e., not failed) at time $T=50$? 
\end{enumerate}
\end{exercise}


\subsection{Coding skills}
\label{sec:progr-assignm}



\begin{exercise}
In this exercise we verify the answers of BH.5.6.5. Read this example first.   

For the python  code below, run it for a small number of sample;  here I choose  \texttt{samples=2} to enable to see the pattern. 
\begin{enumerate}
\item In line 11 we print the value of \texttt{X} in line 10. What is the meaning of \texttt{X}?
\item What is the meaning of \texttt{T} in line 12? 
\item What do we print in line 14? 
\item What is \texttt{expected}? 
\item What is the \texttt{cumsum} of \texttt{expected}? 
\item Now that you understand what is going on, rerun the simulation for a larger number of samples, e.g., 1000.
\end{enumerate}


\begin{minted}[]{python}
import numpy as np
from scipy.stats import expon

np.random.seed(10)

labda = 4
num = 3
samples = 2

X = expon(scale=labda).rvs((samples, num))
print(X)
T = np.sort(X, axis=1)
print(T)
print(T.mean(axis=0))

expected = np.array([labda / ((num - j)) for j in range(num)])
print(expected)
print(expected.cumsum())
\end{minted}
\end{exercise}


\begin{exercise}
Let's check  BH.7.48. Read and solve it first.  

For  the python code below, explain how the small function in lines 6 to 13 works. (You should know that \texttt{x += 1} is an extremely useful abbreviation of the code \texttt{x = x +1}). Then explain the code in lines 25 and 26.

\begin{minted}[]{python}
import numpy as np

np.random.seed(3)


def find_number_of_maxima(X):
    num_max = 0
    M = -np.infty
    for x in X:
        if x > M:
            num_max += 1
            M = x
    return num_max


num = 10
X = np.random.uniform(size=num)
print(X)

print(find_number_of_maxima(X))

samples = 100
Y = np.zeros(samples)
for i in range(samples):
    X = np.random.uniform(size=num)
    Y[i] = find_number_of_maxima(X)

print(Y.mean(), Y.var(), Y.std())
\end{minted}
\end{exercise}


\begin{exercise}
Why is the Exponential Distribution so important? This exercise provides some motivation.


At the Paris metro, a  train arrives every  3  minutes on a platform. Suppose that 50 people arrive between the departure of a train and an arrival.
It seems entirely reasonable to me to model the arrival times of the individual people as distributed on the interval \([0,3]\).
What is the distribution of the inter-arrival times of these people? It turns out to be exponential!  Story BH.13.4.2 provides an explanation (It is not forbidden to read the book beyond what you have to do for this course!); here we use simulation to check this fact. 

For the python code:
\begin{enumerate}
\item Explain the result of line 12.
\item Compare the result of  line 12 and 13;  explain what is \texttt{A[1:]}.
\item Compare the result of  line 12 and 14;  explain what is \texttt{A[:-1}.
\item Explain what is \texttt{X}.
\item Why do I compare $1/\lambda$ and \texttt{X.mean()}?
\item Recall that when $X\sim \Exp{\lambda}$, that $\E X = \sigma (X)$. Hence, what do you expect to see for  \texttt{X.std()}?
\item Run the code for a larger sample, e.g. 50, and discuss (very briefly) your results.
\end{enumerate}


\begin{minted}[]{python}
import numpy as np

np.random.seed(3)


num = 5 # small sample at first, for checking.
start, end = 0, 3
labda = num / (end - start)  # per minute
print(1 / labda)

A = np.sort(np.random.uniform(start, end, size=num))
print(A)
print(A[1:])
print(A[:-1]
X = A[1:] - A[:-1]
print(X)

print(X.mean(), X.std())
\end{minted}

\end{exercise}


\subsection{Challenges}
\label{sec:challenges}

\begin{exercise}

This exercise will given an example of how probability theory can pop up in OR problems, in particular in linear programs. It introduces you to the concept of \textit{recourse models}, which you will learn about in the master course Optimization Under Uncertainty. \textbf{Disclaimer: the story is quite lengthy, but the concepts introduced and questions asked are in fact not very hard. We just added the story to make things more intuitive.}

We consider a pastry shop that only sells one product: chocolate muffins. Every morning at 5:00 a.m., the shop owner bakes a stock of fresh muffins, which he sells during the rest of the day. Making one muffin comes at a cost of $c = \$ 1$ per unit. Any leftover muffins must be discarded at the end of the day, so every morning he starts with an empty stock of muffins.

The owner has one question for you: determine the amount $x$ of muffins that he should make in the morning to minimize his production cost. Note that the owner never wants to disappoint any customer, i.e., he requires that $x \geq d$, where $d$ is the daily demand for muffins.

The problem can be formulated as a linear program (LP):
\begin{align}
    \min_{x \geq 0} \{ cx \ : \ x \geq d \}.
\end{align}
For simplicity, we ignore the fact that $x$ should be integer-valued.

\begin{enumerate}[(a)]
    \item Determine the optimal value $x^*$ for $x$ and the corresponding objective value in case $d$ is deterministic.
\end{enumerate}

Of course, in practice $d$ is not deterministic. Instead, $d$ is a random variable with some distribution. However, note that the LP above is ill-defined if $d$ is a random variable. We cannot guarantee that $x \geq d$ if we do not know the value of $d$. 

You explained the issue to the shop owner and he replies: ``Of course, you're right! You know, whenever I've run out of muffins and a customer asks for one, I make one on the spot. I never disappoint a customer, you know! It does cost me $50 \%$ more money to produce them on the spot, though, you know.''

Mathematically speaking, the shop owner just gave you all the (mathematical) ingredients to build a so-called \textit{recourse model}. We introduce a \textit{recourse variable} $y$ in our model, representing the amount of muffins produced on the spot. Production comes at a unit cost of $q = 1.5 c = \$ 1.5$. Assuming that we know the distribution of $d$, we can then minimize the \textit{expected total cost}:

\begin{align}
    \min_{x \geq 0} \big\{ cx + \E{v(d,x)} \big\},
\end{align}
where $v(d,x)$ is the optimal value of another LP, namely the \textit{recourse problem}:
\begin{align}
    v(d,x) := \min_{y \geq 0} \{ qy \ : \ x + y \geq d \},
\end{align}
for given values of $d$ and $x$. The recourse problem can easily be solved explicitly: we get $y=d-x$ if $d \geq x$ and $y=0$ if $d < x$. So we obtain
\begin{align}
    v(d,x) = q (d - x)^+,
\end{align}
where the operator $(\cdot)^+$ represents the \textit{positive value} operator, defined as
\begin{align}
    (s)^+ = \begin{cases}
    s &\text{if } s \geq 0\\
    0 &\text{if } s < 0.
    \end{cases}
\end{align}

\begin{enumerate}[(a)]
    \setcounter{enumi}{1}
    \item To get some more insight into the model, suppose (for now) that $d \sim U\{10, 20\}$. Solve the model (i.e., find $x^*$). \textit{Hint: First, compute the value of $\E{v(d,x)}$ as a function of $x$. Then find the optimal value of $x$.}
    \item What is the expected recourse cost (expected cost of on-the-spot production) at the optimal solution $x^*$? I.e., compute $\E{v(d,x^*)}$. 
\end{enumerate}

To solve the model correctly, we need the true distribution of $d$. We learn the following from the shop owner: ``My granddaughter, who's always running around in my shop, is a bit data-crazy, you know, so she's been collecting some data. I remember her saying that `the demand from male and female customers are both approximately normally distributed, with mean values both equal to $10$ and standard deviations of $5$'. She also mentioned something about correlation, but I don't remember exactly, you know. It was either almost $1$ or almost $-1$. I hope this helps!''

Mathematically, we've learned that $d = d_m + d_f$, with $(d_m, d_f) \sim \mathcal{N}(\mu, \Sigma)$, where $\mu = (\mu_m, \mu_f) = (10,10)$ and $\Sigma_{11} = \sigma_m^2 = \Sigma_{22}  = \sigma_f^2 = 5^2 = 25$. Finally, $\Sigma_{12} = \Sigma_{21} = Cov(d_m, d_f) = \rho \sigma_m \sigma_f = 25 \rho$. Also, we know that either $\rho \approx 1$ or $\rho \approx -1$.

\begin{enumerate}[(a)]
    \setcounter{enumi}{3}
    \item Calculate $x^*$ and the corresponding objective value for the case $\rho = -1$
    \item Consider the two extreme cases $\rho = 1$ and $\rho = -1$. In which case will the shop owner have lower expected total costs? Explain why. \textit{Hint: you don't have to compute $x^*$ for the case where $\rho = 1$ (this is not easy!). A short, intuitive explanation is sufficient.}
\end{enumerate}


\end{exercise}

\end{document}




\subsection{Measuring inter arrival-times}
\label{sec:orgf820334}

We have a device to measure the time between two arrivals, of jobs for instance, or customers in a shop, or particles in radio-active decay.
After a measurement, the device has to recharge so it cannot measure arrivals that occur within 10 seconds from each other.
Also, if there is no arrival within 50, it resets itself.
Hence, the device cannot easily measure inter-arrival times longer than 1 minute. Assume that the inter-arrival times are exponentially distributed with some unknown \(\lambda\). Let \(\bar x = \sum_{n=1}^N x_{n} /N\) be the sample mean of \(N\) measured inter-arrival times. 

\begin{enumerate}
\item Explain that, if \(\lambda \ll 1\) minute,  \(\hat \lambda = \bar x - 10\) is a reasonable estimator for \(\lambda\).
\item How would you obtain a reasonable estimate of \(\lambda\) when \(\lambda\) \(\gg\) 1 minute?
\end{enumerate}



\begin{todo}
Let $X\sim\Exp{\lambda}$ and $Y \sim N(\mu, \sigma)$, independent of $X$.
So, draw $X$ first and let the outcome be $x$; then draw $Y\sim N(x, \sigma)$.
Take $\lambda=4$, $\mu = 5$, $\sigma=3$.
\begin{enumerate}
\item Make a 3D plot of $F_{X,Y}$.
\item Make a 3D plot of $f_{X,Y}$.
\item Plot $f_{X}$, i.e., plot $\mu e^{-\mu t}$.
  Then use simulation to marginalize out $Y$ from $f_{X,Y}$ to obtain $\hat f_X$; we write $\hat f_X$ because it has been obtained from simulation.
  Plot $\hat f_X$ in the same figure as $f_X$, and compare the result.
\item Use simulation to estimate $f_{X|Y}$. Plot this in the same graph for various values of $X=x$. 
\item Make a 3D plot of $f$
\end{enumerate}
\end{todo}

\begin{todo}
Let $X\sim\Exp{\lambda}$ and $Y| X \sim N(X, \sigma)$. So,  draw $X$ first and let the outcome be $x$;  then draw $Y\sim N(x, \sigma)$. Take $\lambda = 5$, $\sigma=3$. 
\begin{enumerate}
\item Use simulation to estimate $\E Y$. 
\item Make a 3D plot of $F_{X,Y}$.
\item Make a 3D plot of $f_{X,Y}$.
\item Plot $f_{X}$, i.e., plot $\lambda e^{-\lambda t}$.
  Then use simulation to marginalize out $Y$ from $f_{X,Y}$ to obtain $\hat f_X$.
  Plot $\hat f_X$ in the same figure as $f_X$, and compare the result.
\item Use simulation to estimate $f_{X|Y}$. Plot this in the same graph for various values of $X=x$. 
\item Make a 3D plot of $f$
\end{enumerate}
\end{todo}


