\documentclass[assignments]{subfiles}

\begin{document}


\section{Assignment 3, TBD}
\label{sec:assignment-3}

Topics of chapter 8.

\subsection{Have you read well?}
\label{sec:have-you-read-1}

\begin{exercise}
Explain in your own words:
\begin{enumerate}
\item What is a prior?
\item What is a conjugate prior?
\end{enumerate}
\end{exercise}

\begin{exercise}
Look up on the web: what is the conjugate prior of the multinomial distribution?
\end{exercise}



\begin{exercise} The  baker of Poincare. TBD


Henri Poincaré was a French mathematician who taught at the Sorbonne around 1900. The following anecdote about him is probably fabricated, but it makes an interesting probability problem.
Supposedly Poincaré suspected that his local bakery was selling loaves of bread that were lighter than the advertised weight of 1 kg, so every day for a year he bought a loaf of bread, brought it home and weighed it. At the end of the year, he plotted the distribution of his measurements and showed that it fit a normal distribution with mean 950 g and standard deviation 50 g. He brought this evidence to the bread police, who gave the baker a warning.
For the next year, Poincaré continued the practice of weighing his bread every day. At the end of the year, he found that the average weight was 1000 g, just as it should be, but again he complained to the bread police, and this time they fined the baker.


Why? Because the shape of the distribution was asymmetric. Unlike the normal distribution, it was skewed to the right, which is consistent with the hypothesis that the baker was still making 950 g loaves, but deliberately giving Poincaré the heavier ones.
Exercise 5.6 Write a program that simulates a baker who chooses n loaves from a distribution with mean 950 g and standard deviation 50 g, and gives the heaviest one to Poincaré. What value of n yields a distribution with mean 1000 g? What is the standard deviation?

\end{exercise}


\subsection{About exam level}
\label{sec:chapter-8}

\begin{exercise}
Let $X, Y$ iid $\sim \Unif{[0,1]}$.
\begin{enumerate}
\item What is the joint CDF of $X+Y, XY$?
\item What is the joint PDF of $X+^Y XY$?
\item Compute $\P{X+Y\leq 1, XY\leq 2/9}$.
\end{enumerate}
\end{exercise}

\subsection{Coding skills}
\label{sec:coding-skills-1}


\begin{exercise}\label{ex:2}
How many ping pong balls fit into an Airbus Beluga?
One way to answer this is as follows.
According to this \href{https://en.wikipedia.org/wiki/Airbus\_Beluga}{wikipage} the cargo volume $V$ of this airplane is $1500 \m^{3}$.
But this number is based on the physical dimensions that is available to store containers, tanks, and so on.
So, I estimate the volume as about twice that amount, i.e., $V = 2500 \m^{3}$.
The volume of a ping pong ball is $v = 4 \pi r^3/3  = \py{4*3.14*8/3} \cm^{3}$ with $r=2$ cm.
A plain division gives \py{2500/33.5} ping pong balls.
Note, I left out the $10^{6}$ conversion from meters to cm, and I do not take into the sphere packing factor.
Besides that, I hope you agree with me that providing an result with the precision as given here is plain ridiculous.
(But from reason incomprehensible to me, even professional econometricians like to report results with 10 digits or more, without questioning the precision.)


However, I know that the volumes of an air plane and a ping pong ball is an estimate, rather than a precise number as assumed above.
It seems to be better to approximate $V$ and $v$ as rvs.
Let's assume that
   \begin{align*}
V & \sim N(2500, 500^{2}), & v  & \sim N(33.5, \sigma=0.5^{2}),
\end{align*}
where the variances express my trust in my guess work.
What is now the mean of $N = V/v$ and its std?
In fact, finding the closed form expression for the distribution of $N$ is not entirely simple.
However, with simulation it's easy to get an estimate.

\begin{enumerate}
\item How does this exercise relate to BH.8.11 and BH.8.12? What is similar, what are crucial differences?
\item Use the documentation of the \texttt{norm} (\texttt{rnorm})function of python (R) to explain we we set the scale as we do.
  Relate this to location-scale discussion in BH.
\item  Explain lines 8 and 11 of the python code or lines 5 and 8 of the R code. (If you are interested in both languages, also comment on the difference)
\item Use the code below to provide these estimates.
\item Contrary to BH.7.1.25 if you run the code below, you'll see that $\E N < \infty$, and, in fact, very near to the deterministic answer.
But isn't this strange?
We divide two normal random variables, just like BH.7.1.25, but there the expectation is infinite.
Comment on the difference.
\end{enumerate}


The numerical results suggest the interesting guess $\V N \approx \V V * \V v$, but  is this true more generally? In~\cref{ex:3} we research this problem in more detail.

\begin{minted}[]{python}
import numpy as np
from scipy.stats import norm

num = 500

np.random.seed(3)

V = norm(loc=2500, scale=500)
v = norm(loc=33.5, scale=0.5)

print(V.mean(), V.std()) # just a check

N = V.rvs(num) / v.rvs(num)
print(N.mean(), N.std())

print(2500/33.5)
print(np.sqrt(500*0.5))
\end{minted}

\begin{minted}[]{R}
num <- 500

set.seed(3)

V = rnorm(num, 2500, 500)
v = rnorm(num, 33.5, 0.5)

N = V / v
paste(mean(N), sd(N))

2500/33.5
sqrt(500*0.5)
\end{minted}
\end{exercise}

\begin{exercise}
We start from BH.8.27, and we are interested in the difference between the distribution of $X+Y+Z$ and the normal distribution.

\begin{solution}
\end{solution}
\end{exercise}




\end{document}


\subsection{Challenges}
\label{sec:challenges-1}



\begin{exercise}\label{ex:3}
This is a continuation of~\cref{ex:2}.  TBD.

\begin{verbatim}
Als V[V] = 0 of V[v] = 0, is rechts 0, links niet perse.
Maar er is nog een betere reden waarom dit niet kan kloppen. Definieer V' = 2V, v'=2v en N' = V'/v'. Dan N=N', dus links blijft gelijk. Maar rechts wordt 16 keer zo groot...

Ik denk dat ik het heb opgelost. Merk op dat we aannemen dat V en v onafhankelijk zijn. Dus
V[V/v] = E[V^2/v^2] - E[V/v]^2 = E[V^2]E[1/v^2] - E[V]^2E[1/v]^2 = V[V]V[1/v] + E[V]^2 E[1/v^2]  + E[V^2] E[1/v]^2 - 2 E[V]^2E[1/v]^2= V[V]V[1/v] + E[V]^2 V[1/v] + V[V] E[1/v]^2.
Als we er van uit gaan dat we V en v voldoende precies weten, geldt er V[V]V[1/v] << E[V]^2 V[1/v] + V[V] E[1/v]^2.
De eerste-orde Taylorbenadering van 1/v rond E[v] is 2/E[v] - v/E[v]^2. We kunnen dus benaderen V[1/v] \approx  V[v]/E[v]^4.
Dezelfde benadering geeft ook E[1/v]^2 \approx 1/ E[v]^2. Dus het resultaat is al met al  V[V/v]\approx  E[V]^2 V[v]/E[v]^4 + V[V] / E[v]^2 = E[V/v]^2( V[v]/E[v]^2 +  V[V]/E[V]^2 ).
Ofwel: V[V/v]/E[V/v]^2=  V[v]/E[v]^2 +  V[V]/E[V]^2, dus SCV(V/v) \approx SCV(V)  + SCV(v).
Overigens: de benadering zit hier in SCV(1/v) \approx SCV(v). Immers, SCV(V/v) = SCV(V)  + SCV(1/v) wegens onafhankelijkheid.

\end{verbatim}


\end{exercise}
