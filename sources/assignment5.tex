\documentclass[assignments]{subfiles}

\begin{document}


\section{Assignment 5, TBD}

\subsection{Have you read well?}


\begin{exercise}
Use Eve's law to show that $\V Y \geq \V{\E{Y\given X}}$.
\end{exercise}


\subsection{Exercises at about exam level}
\label{sec:exercises-at-about-1}

We derive Eve's law in a slightly different way then in BH.

Define $hat X = \E{X\given Y}$ as an \emph{estimator} of $X$ and $\tilde X = X - \hat X$ as the estimation error.

\begin{exercise}\label{ex:5}
Show that $\E{\tilde X\given Y} = 0$.
\begin{solution}
\begin{align}
\label{eq:4}
\E{\tilde X \given Y} = \E{X - \hat X \given Y} &=
\E{X\given Y}  - \E{ \E{X\given Y}\given Y}  \\
&=\E{X\given Y}  -  \E{X\given Y}\E{1\given Y}  \\
&=\E{X\given Y}  -  \E{X\given Y}1 = 0
\end{align}
\end{solution}
\end{exercise}

\begin{exercise}\label{ex:4}
Prove that $\E{\tilde X} = 0$. What does it mean that $\E{\tilde X} =0$?
\begin{hint}
  Use~\cref{ex:5}
\end{hint}
\begin{solution}
\begin{equation}
\label{eq:5}
\E{\tilde X} = \E{\E{\tilde X\given Y}} = \E{ 0 \given Y} =0.
\end{equation}
This means that the estimation error $\tilde X$ does not have bias.
\end{solution}
\end{exercise}

\begin{exercise}
Prove that $\E{\tilde X\hat X} = 0$.
\begin{hint}
  Use~\cref{ex:4} and the definitions.
\end{hint}
\begin{solution}
\begin{align}
\label{eq:6}
\E{\tilde X\hat X} &= \E{\E{\tilde X \hat X \given Y}}  \\
&= \E{\E{\tilde X \E{X\given Y} \given Y}}  \\
&= \E{\E{X\given Y}\E{\tilde X \given Y}}  \\
&= \E{\E{X\given Y} 0  \given Y}  = 0
\end{align}
\end{solution}
\end{exercise}

\begin{exercise}
Show that $\cov{\hat X, \tilde X} = 0$. Conclude that
\begin{equation}
\label{eq:8}
\V{X} = \V{\hat X + \tilde X} = \V{\hat X} + \V{\tilde X}.
\end{equation}
\begin{solution}
Using the previous exercises,
\begin{equation}
\label{eq:7}
\cov{\hat X, \tilde X} = \E{\hat X \tilde X} - \E{\hat X} \E{\tilde X} = 0 - \E{\hat X} 0 = 0.
\end{equation}
Next, from the definition of $\tilde X = X - \hat X \implies X = \tilde X + \hat X$.
The variance of the sum is the sum of the variances since $\hat X$ and $\tilde X$ are uncorrelated.
\end{solution}
\end{exercise}

\begin{exercise}
Prove that $\V{\tilde X} = \E{\V{X\given Y}}$. Conclude Eve's law.
\begin{solution}
Since $\E{\tilde X} = 0$,
\begin{align}
\label{eq:9}
\V{\tilde X} &= \E{\tilde X^{2}}  \\
&= \E{\E{\tilde X^{2}\given Y}} \\
&= \E{\E{(X - \hat X)^{2}\given Y}} \\
&= \E{\E{(X - \E{X\given Y})^{2}\given Y}} \\
&= \E{\V{X\given Y}},
\end{align}
where the last equation follows by the definition of $\V{X\given Y}$.

For Eve's law, use the above and the previous exercise to see that
\begin{equation}
\label{eq:10}
\V{X} = \V{\hat X} + \V{\tilde X} = \V{\E{X\given Y}} + \E{\V{X\given Y}}.
\end{equation}
\end{solution}
\end{exercise}



\section{Bayes' Billiards}
\label{sec:orgf93559e}
Take \(n=100\) samples.

\subsection{One coin}
\label{sec:org1e9a031}
\begin{enumerate}
\item Take success probability \(p=1/2\).
\item Make matrix with \(n\) rows, \(n\) columns. Each row is an experiment of \(n\) throws of the coin.
\item Plot the histogram of the number of heads.
\end{enumerate}

\subsection{Three coins}
\label{sec:orga657159}
\begin{enumerate}
\item Take three coins with success probabilities \(p=1/4, 1/2, 3/4\).
\item Make a simulation for each coin.
\item If we select a coin with probability \(1/3\), the total histogram is the 1/3 times the sum of the histograms of each of the coins. That is, \(\P{X=k} = \P{X=k|C = i}\P{C=i}\), where \(C\) is one of chosen coins;  here we take \(\P{C=i} = 1/3\).
\end{enumerate}

\subsection{five coins}
\label{sec:orga1a3b28}
\begin{enumerate}
\item Select with uniform probability one out of five coins with success probabilities \(p=i/5\),  \(i=1,\ldots, 5\).
\item Make a simulation for each coin.
\item Make the histogram \(\P{X=k} = \P{X=k|C = i}\P{C=i}\), where \(\P{C=i} = 1/5\).
\end{enumerate}


\subsection{Coding skills}
\label{sec:coding-skills-1}

\paragraph{The mystery box}

We use  simulation to solve  BH.9.7.
Read it now, i.e., before reading the text below, then read the code below.
Note how short  this code is;  amazing, isn't it?



\begin{minted}[]{python}
import numpy as np
from scipy.stats import uniform
import matplotlib.pyplot as plt

np.random.seed(3)


N = 1000
a, b = 0, 1000_000
V = uniform(a, b).rvs(N)

x_range = np.linspace(b / 5, b / 2, num=50)
y = np.zeros_like(x_range)

for i, b in enumerate(x_range):
    payoff = (V - b) * (b >= V / 4)
    y[i] = payoff.mean()


plt.plot(x_range, y)
plt.show()
\end{minted}


\begin{minted}[]{R}

\end{minted}


\begin{exercise}
Use the scipy documentation to explain why $V\sim\Unif{[0,10^{6}]}$.
\begin{solution}
\end{solution}
\end{exercise}



\begin{exercise}
What are the smallest and the largest value of \verb|x_range|?
\begin{solution}
\end{solution}
\end{exercise}

\begin{exercise}
Run the code above and make a graph. Include the graph in your report, and explain what you see in the graph. For instance, is there a maximum? If so, can you explain where the maximum occurs? Can you explain how the maximum should be?
\begin{solution}
\end{solution}
\end{exercise}


\begin{exercise}
Suppose after seeing the graph of the payoffs, and this graph would only increase, or decrease, how would you change \verb|x_range|? Do you expect to see a maximum?
\begin{solution}
\end{solution}
\end{exercise}




\begin{exercise}
For N small, e.g. N=10, you can get quite strange values. Why is that?
\begin{solution}
\end{solution}
\end{exercise}


\begin{exercise}
Change the acceptance threshold from to $V/4$ to $V/5$ (or $V/6$, or some other value you like), and make a graph of the payoffs.
Include the graph in your report.
\begin{solution}
\end{solution}
\end{exercise}

\begin{exercise}
Change the payoff function to e.g $\sqrt{V-b}$, or some weird function that you like particularly such as $\sin |V-b|$ (any non-trivial function goes).
Make a graph of the  mean and std of the payoff. Can you explain your graph?
\begin{solution}
\end{solution}
\end{exercise}

\paragraph{Kelly makes bets}

This simulation exercise is based on BH.9.25.  Please read the exercise first, and then the code below.



\begin{minted}[]{python}
import numpy as np
from scipy.stats import bernoulli

np.random.seed(3)

n = 5
num = 10

p = 0.5
S = bernoulli(p).rvs([num, n]) * 2 - 1
# print(S)

x = np.zeros([num, n])
x[:, 0] = 100
# print(x)
f = 0.25

for i in range(1, n):
    x[:, i] = x[:, i - 1] * (1 + f * S[:, i])

# print(x)
print(x.mean(axis=0), x.std(axis=0))
\end{minted}

\begin{exercise}
The documentation of \texttt{bernoulli} tells us that we get an array (matrix) with \texttt{num} rows and $n$ columns with $0$ and $1$s. Explain that by multiplying with 2 and subtracting 1 we get an array with $1$s and $-1$s.
\begin{solution}
\end{solution}
\end{exercise}

\begin{exercise}
The $j$ th columns of \texttt{X} corresponds to the $j$th bet. Explain how line 19 of the python code works.
\begin{solution}
\end{solution}
\end{exercise}

\begin{exercise}
Choose some $p$ and $f$ to your liking, run an experiment, and compare the values of the experiment(simulation) to the theoretical values, i.e., what you get when you solve  problem BH.9.25.
\begin{solution}
\end{solution}
\end{exercise}






\end{document}
