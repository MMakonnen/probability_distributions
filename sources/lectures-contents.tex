\author{Nicky van Foreest and Ruben van Beesten}

\opt{all-solutions-at-end}{
\Opensolutionfile{hint}
\Opensolutionfile{ans}
}


\begin{document}
\maketitle
\tableofcontents

\section{Lecture 1}

\begin{exercise}
Consider 12 football players on a football field.
Eleven of them are players of FC Barcelone, the other one is an arbiter.
We select a random player, uniform.
This player must take a penalty.
The probability that a player of Barcelone scores is 70\%, for the arbiter it is 50\%.
Let $P\in \{A, B\}$ be r.v that corresponds to the selected player, and $S\in\{0,1\}$ be the score.
\begin{enumerate}
\item What is the PMF? In other words, determine $\P{P = B, S=1}$ and so on for all possibilities.
\item What is $\P{S=1}$? What is $\P{P=B}$?
\item Show that $S$ and $P$ are dependent.
\end{enumerate}
\begin{solution}
Here is the joint PMF:
\begin{align}
  \label{}
\P{P=A, S=1} &= \frac{1}{12}0.5 & \P{P=A, S=0} &= \frac{1}{12}0.5 \\
\P{P=B, S=1} &= \frac{11}{12}0.7 & \P{P=B, S=0} &= \frac{11}{12}0.3.
\end{align}
Now the marginal PMFs
\begin{align*}
\P{S=1}  &= \P{P=A, S=1} + \P{P=B, S=1} = 0.042 + 0.64 = 0.683 = 1-\P{S=0}\\
\P{P=B}  &= \frac{11}{12} = 1-\P{P=A}.
\end{align*}
For independence we take the definition.
In general, for all outcomes $x,y$ we must have that $\P{X=x, Y=y} = \P{X=x}\P{Y=y}$.
For our present example, let's check for  a particular outcome:
\begin{align*}
\P{P=B, S=1} &= \frac{11}{12}\cdot0.7 \neq \P{P=B}\P{S=1} = \frac{11}{12} \cdot 0.683
\end{align*}
The joint PMF is obviously not the same as the product of the marginals, which implies that $P$ and $S$ are not independent.
\end{solution}
\end{exercise}



An insurance company receives on a certain day two claims $X, Y \geq 0$.
We will find the PMF of the loss $Z=X+Y$ under different assumptions.

The joint CDF $F_{X,Y}$ and joint PMF $p_{X,Y}$ are assumed known.

\begin{exercise}
Why is it not interesting to consider the case $\{X=0, Y=0\}$?
\begin{solution}
When the claim sizes are $0$, then the insurance company does not receive a claim.
\end{solution}
\end{exercise}


\begin{exercise}
Find an expression for the PMF of $Z=X+Y$.
\begin{solution}
By the fundamental bridge,
\begin{align}
  \label{}
\P{Z=k}
&= \sum_{i,j} \1{i+j=k} p_{X,Y}(i,j) \\
&= \sum_{i,j} \1{i, j \geq 0} \1{j=k-i} p_{X,Y}(i,j) \\
&= \sum_{i=0}^{k} p_{X,Y}(i,k-i).
\end{align}
\end{solution}
\end{exercise}

Suppose $p_{X,Y}(i,j) = c \1{i=j}\1{1\leq i \leq 4}$.


\begin{exercise}
What is $c$?
\begin{solution}
$c=1/4$ because there are just four possible values for $i$ and $j$.
\end{solution}
\end{exercise}

\begin{exercise}
What is $F_{X}(i)$?
What is $F_{Y}(j)$?
\begin{solution}
Use marginalization:
\begin{align}
F_X(k) &=  F_{X,Y}(k, \infty ) = \sum_{i\leq k} \sum_j p_{X,Y}(i,j) \\
 &= \frac{1}{4}\sum_{i\leq k} \sum_j \1{i=j}\1{1\leq i \leq 4}\\
 &= \frac{1}{4}\sum_{i\leq k} \1{1\leq i \leq 4} \\
&=k/4,\\
F_Y(j) &= j/4.
\end{align}
\end{solution}
\end{exercise}


\begin{exercise}
Are $X$ and $Y$ dependent?  If so, why, because $1=F_{X,Y}(4,4)= F_X(4)F_Y(4)$?
\begin{solution}
  The equality in the question must hold for all $i,j$, not only for $i=j=4$.
  If you take $i=j=1$, you'll see immediately that $F_{X,Y}(1,1)\neq F_X(1)F_Y(1)$:
  \begin{align}
    \label{eq:23}
    \frac{1}{4} = F_{X,Y}(1,1) \neq F_{X}(1) F_Y(1) = \frac{1}{4}\frac{1}{4}.
  \end{align}
\end{solution}
\end{exercise}

\begin{exercise}
What is $\P{Z=k}$?
\begin{solution}
$\P{Z=2} = \P{X=1, Y=1} = 1/4 = \P{Z=4}$, etc.
$\P{Z=k} = 0$ for $k\not \in \{2, 4, 6, 8\}$.
\end{solution}
\end{exercise}


\begin{exercise}
What is $\V Z$?
\begin{solution}
Here is one approach
\begin{align}
\label{eq:3}
\V Z &= \E{Z^2} - (\E Z)^{2}\\
\E{Z^2} &= \E{(X+Y)^{2}} = \E{X^{2}} + 2\E{XY} + \E{Y^{2}} \\
(E{Z})^{2} &= (\E X + \E Y)^{2} \\
 &= (\E X)^2 + 2\E X \E Y + (\E Y)^{2} \\
&\implies \\
\V Z &= \E{Z^2} - (\E Z)^{2}\\
 &= \V X + \V Y + 2 (\E{XY} - (\E X \E Y))\\
\E{XY} &= \sum_{ij} ijp_{X,Y}(i,j) = \frac 1 4 (1 + 4 + 9 + 16) = \ldots \\
\E{X^{2}} &= \ldots
\end{align}
The numbers are for you to compute.
\end{solution}
\end{exercise}


Now take $X, Y$ iid $\sim\Unif{\{1,2,3,4\}}$ (so now no longer $p_{X,Y}(i,j) \neq \1{i=j}\1{1\leq i \leq 4}$).

\begin{exercise}
What is $\P{Z=4}$?
\begin{solution}
\begin{align}
\label{eq:2}
\P{Z=4}
&= \sum_{i, j} \1{i+j=4} p_{X,Y}(i,j) \\
&= \sum_{i=1}^4 \sum_{j=1}^{4} \1{j=4-i} \frac{1}{16} \\
&= \sum_{i=1}^3  \frac{1}{16} \\
&= \frac{3}{16}.
\end{align}
\end{solution}
\end{exercise}

\begin{remark}
We can  make lots of variations on this theme.
\begin{enumerate}
\item Let $X\in \{1,2,3\}$ and $Y\in \{1,2,3,4\}$.
\item Take $X\sim\Pois{\lambda}$ and $Y\sim\Pois{\mu}$. (Use the chicken-egg story)
\item We can make $X$ and $Y$ such that they are (both) continuous, i.e., have densities.
  The conceptual ideas\footnote{Unless you start digging deeper.
    Then things change drastically, but we skip this technical stuff.}
  don't change much, except that the summations become integrals.
\item Why do people often/sometimes (?)
  model the claim sizes as iid $\sim\Norm{\mu, \sigma^{2}}$?
  There is a slight problem with this model (can real claim sizes be negative?), but what is the way out?
\item The example is more versatile than you might think. Here is another interpretation.

A supermarket has 5 packets of rice on the shelf.
Two customers buy rice, with amounts $X$ and $Y$.
What is the probability of a lost sale, i.e., $\P{X+Y>5}$?
What is the expected amount lost, i.e., $\E{ \max{X+Y - 5,0}}$?

Here is yet another.
Two patients arrive in to the first aid of a hospital.
They need $X$ and $Y$ amounts of service, and there is one doctor.
When both patients arrive at 2 pm, what is the probability that the doctor has work in overtime (after 5 pm), i.e., $\P{X+Y > 5- 2}$?
\end{enumerate}
\end{remark}


\section{Lecture 2}

Read the problems of \verb|memoryless\_excursions.pdf|.
All the problems in that document relate to topics discussed in Sections BH.7.1 and BH.7.2, and quite a lot of topics you have seen in the previous course on probability theory.


\section{Lecture 3}

\begin{enumerate}
\item Why don't you setup a github repo to collaborate on making  solutions of the book?
\item There is a \LaTeX template on github to help you get started with the assignment.
\item
\end{enumerate}

\begin{exercise}
We ask a married woman on the street her height $X$. What does this tell us about the height $Y$ of her spouse? We suspect that taller/smaller people choose  taller/smaller partners, so, given $X$, a simple estimator $\hat Y$ of $Y$ is given by
\begin{equation*}
  \hat Y = a X + b.
\end{equation*}
But how to determine $a$ and $b$? A common method to find $a$ and $b$ such that the function
\begin{equation*}
  f(a,b) = \E{(Y-\hat Y)^2}
\end{equation*}
is minimized. Show that the optimal values are such that
\begin{align*}
  \hat Y = \E Y + \rho \V Y (X - \E X),
\end{align*}
where $\rho$ is the correlation between $X$ and $Y$.

\begin{solution}
We take the partial derivatives of $f$ with respect to $a$ and $b$, and solve for $a$ and $b$.
  \begin{align*}
f(a,b) &= \E{(Y-\hat Y)^2} \\
 &= \E{(Y-a X - b)^2} \\
 &= \E{Y^{2}} - 2a\E{YX} - 2b\E Y + a^{2}\E{X^2} + 2 ab \E X + b^{2}\\
\partial_{a} f &=-2 \E{YX} + 2a \E{X^2} + 2 b \E X = 0 \\
&\implies a \E{X^2} =  \E{YX}  -  b \E X \\
\partial_{b} f &=-2 \E{Y}  + 2 a \E X  + 2 b= 0 \\
& \implies  b = \E Y - a \E{X}\\
a \E{X^2} &=  \E{YX}  -  \E X (\E Y - a \E X) \\
&\implies  a (\E{X^{2}} - \E X \E X)  = \E{YX} - \E X \E Y  \\
&\implies a = \frac{\cov{X,Y}}{\V X} = \rho \V Y\\
b &= \E Y - \rho \V Y \E X\\
\hat Y &= a X + b \\
&= \rho \V Y X + \E Y - \rho \V Y \E X \\
&= \rho \V Y (X-\E X)  + \E Y.
  \end{align*}
What a neat formula! Memorize the derivation, at least the structure. You'll come across many more optimization problems.

What if $\rho=0$?
\end{solution}
\end{exercise}


\begin{exercise}
$N$ people throw their hat in a box. After shuffling, each of them takes out a hat at random. How many people do you expect to take out their own hat (i.e., the hat they put in the box); what is the variance? In BH.7.46 you have to solve this analytically. In the exercise here you have to write a simulator for compute the expectation and variance.
\begin{solution}
Let us first do one run.
\begin{pyblock}[][numbers=left,frame=lines]
import numpy as np
from numpy.random import uniform

np.random.seed(3)

N = 4
X = np.arange(N)
np.random.shuffle(X)
print(X)
print(np.arange(N))
print((X == np.arange(N)))
print((X == np.arange(N)).sum())
\end{pyblock}
Here are the results of the print statements: $X = \py{X}$. The matches are \py{(X == np.arange(N))}; we see that $X[1] = 1$ (recall, python arrays start at index 0, not at 1, so $X[1]$ is the second element of $X$, not the first), so that the second person picks his own hat. The number of matches is therefore 1 for this simulation.

Now put the people to work, and let them pick hats for $50$ times.
\begin{pyblock}[][numbers=left,frame=lines]
import numpy as np
from numpy.random import uniform

np.random.seed(3)

num_samples = 50
N = 5

res = np.zeros(num_samples)
for i in range(num_samples):
    X = np.arange(N)
    np.random.shuffle(X)
    res[i] = (X == np.arange(N)).sum()

print(res.mean(), res.var())
\end{pyblock}
Here is the number of matches for each round: \py{res}
The mean and variance are as follows: $\E X = \py{res.mean()}$ and $\V X = \py{res.var()}$.

For your convenience, here's the R code
\begin{minted}[]{R}
# set seed such that results can be recreated
set.seed(42)

# number simulations and people
numSamples <- 50
N <- 5

# initialize empty result vector
res <- c()

# for loop to simulate repeatedly
for (i in 1:numSamples) {

  # shuffle the N hats
  x <- sample(1:N)

  # number of people picking own hat (element by element the vectors x and
  # 1:N are compared, which yields a vector of TRUE and FALSE, TRUE = 1 and
  # FALSE = 0)
  correctPicks <- sum(x == 1:N)

  # append the result vector by the result of the current simulation
  res <- append(res, correctPicks)
}

# printing of observed mean and variance
print(mean(res))
print(var(res))
\end{minted}


\end{solution}
\end{exercise}



\opt{all-solutions-at-end}{
\clearpage
\Closesolutionfile{ans}
\section{Solutions}
\input{ans}
}

\end{document}

%%% Local Variables:
%%% TeX-master: "lectures-demo"
%%% End:
