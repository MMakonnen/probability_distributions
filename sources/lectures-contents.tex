\author{Nicky van Foreest and Ruben van Beesten}

\opt{all-solutions-at-end}{
\Opensolutionfile{hint}
\Opensolutionfile{ans}
}


\begin{document}
\maketitle
\tableofcontents

\section{Lecture 1}

\begin{exercise}
Consider 12 football players on a football field.
Eleven of them are players of FC Barcelone, the other one is an arbiter.
We select a random player, uniform.
This player must take a penalty.
The probability that a player of Barcelone scores is 70\%, for the arbiter it is 50\%.
Let $P\in \{A, B\}$ be r.v that corresponds to the selected player, and $S\in\{0,1\}$ be the score.
\begin{enumerate}
\item What is the PMF? In other words, determine $\P{P = B, S=1}$ and so on for all possibilities.
\item What is $\P{S=1}$? What is $\P{P=B}$?
\item Show that $S$ and $P$ are dependent.
\end{enumerate}
\begin{solution}
Here is the joint PMF:
\begin{align}
  \label{}
\P{P=A, S=1} &= \frac{1}{12}0.5 & \P{P=A, S=0} &= \frac{1}{12}0.5 \\
\P{P=B, S=1} &= \frac{11}{12}0.7 & \P{P=B, S=0} &= \frac{11}{12}0.3.
\end{align}
Now the marginal PMFs
\begin{align*}
\P{S=1}  &= \P{P=A, S=1} + \P{P=B, S=1} = 0.042 + 0.64 = 0.683 = 1-\P{S=0}\\
\P{P=B}  &= \frac{11}{12} = 1-\P{P=A}.
\end{align*}
For independence we take the definition.
In general, for all outcomes $x,y$ we must have that $\P{X=x, Y=y} = \P{X=x}\P{Y=y}$.
For our present example, let's check for  a particular outcome:
\begin{align*}
\P{P=B, S=1} &= \frac{11}{12}\cdot0.7 \neq \P{P=B}\P{S=1} = \frac{11}{12} \cdot 0.683
\end{align*}
The joint PMF is obviously not the same as the product of the marginals, which implies that $P$ and $S$ are not independent.
\end{solution}
\end{exercise}



An insurance company receives on a certain day two claims $X, Y \geq 0$.
We will find the PMF of the loss $Z=X+Y$ under different assumptions.

The joint CDF $F_{X,Y}$ and joint PMF $p_{X,Y}$ are assumed known.

\begin{exercise}
Why is it not interesting to consider the case $\{X=0, Y=0\}$?
\begin{solution}
When the claim sizes are $0$, then the insurance company does not receive a claim.
\end{solution}
\end{exercise}


\begin{exercise}
Find an expression for the PMF of $Z=X+Y$.
\begin{solution}
By the fundamental bridge,
\begin{align}
  \label{}
\P{Z=k}
&= \sum_{i,j} \1{i+j=k} p_{X,Y}(i,j) \\
&= \sum_{i,j} \1{i, j \geq 0} \1{j=k-i} p_{X,Y}(i,j) \\
&= \sum_{i=0}^{k} p_{X,Y}(i,k-i).
\end{align}
\end{solution}
\end{exercise}

Suppose $p_{X,Y}(i,j) = c \1{i=j}\1{1\leq i \leq 4}$.


\begin{exercise}
What is $c$?
\begin{solution}
$c=1/4$ because there are just four possible values for $i$ and $j$.
\end{solution}
\end{exercise}

\begin{exercise}
What is $F_{X}(i)$?
What is $F_{Y}(j)$?
\begin{solution}
Use marginalization:
\begin{align}
F_X(k) &=  F_{X,Y}(k, \infty ) = \sum_{i\leq k} \sum_j p_{X,Y}(i,j) \\
 &= \frac{1}{4}\sum_{i\leq k} \sum_j \1{i=j}\1{1\leq i \leq 4}\\
 &= \frac{1}{4}\sum_{i\leq k} \1{1\leq i \leq 4} \\
&=k/4,\\
F_Y(j) &= j/4.
\end{align}
\end{solution}
\end{exercise}


\begin{exercise}
Are $X$ and $Y$ dependent?  If so, why, because $1=F_{X,Y}(4,4)= F_X(4)F_Y(4)$?
\begin{solution}
  The equality in the question must hold for all $i,j$, not only for $i=j=4$.
  If you take $i=j=1$, you'll see immediately that $F_{X,Y}(1,1)\neq F_X(1)F_Y(1)$:
  \begin{align}
    \label{eq:23}
    \frac{1}{4} = F_{X,Y}(1,1) \neq F_{X}(1) F_Y(1) = \frac{1}{4}\frac{1}{4}.
  \end{align}
\end{solution}
\end{exercise}

\begin{exercise}
What is $\P{Z=k}$?
\begin{solution}
$\P{Z=2} = \P{X=1, Y=1} = 1/4 = \P{Z=4}$, etc.
$\P{Z=k} = 0$ for $k\not \in \{2, 4, 6, 8\}$.
\end{solution}
\end{exercise}


\begin{exercise}
What is $\V Z$?
\begin{solution}
Here is one approach
\begin{align}
\label{eq:3}
\V Z &= \E{Z^2} - (\E Z)^{2}\\
\E{Z^2} &= \E{(X+Y)^{2}} = \E{X^{2}} + 2\E{XY} + \E{Y^{2}} \\
(E{Z})^{2} &= (\E X + \E Y)^{2} \\
 &= (\E X)^2 + 2\E X \E Y + (\E Y)^{2} \\
&\implies \\
\V Z &= \E{Z^2} - (\E Z)^{2}\\
 &= \V X + \V Y + 2 (\E{XY} - (\E X \E Y))\\
\E{XY} &= \sum_{ij} ijp_{X,Y}(i,j) = \frac 1 4 (1 + 4 + 9 + 16) = \ldots \\
\E{X^{2}} &= \ldots
\end{align}
The numbers are for you to compute.
\end{solution}
\end{exercise}


Now take $X, Y$ iid $\sim\Unif{\{1,2,3,4\}}$ (so now no longer $p_{X,Y}(i,j) \neq \1{i=j}\1{1\leq i \leq 4}$).

\begin{exercise}
What is $\P{Z=4}$?
\begin{solution}
\begin{align}
\label{eq:2}
\P{Z=4}
&= \sum_{i, j} \1{i+j=4} p_{X,Y}(i,j) \\
&= \sum_{i=1}^4 \sum_{j=1}^{4} \1{j=4-i} \frac{1}{16} \\
&= \sum_{i=1}^3  \frac{1}{16} \\
&= \frac{3}{16}.
\end{align}
\end{solution}
\end{exercise}

\begin{remark}
We can  make lots of variations on this theme.
\begin{enumerate}
\item Let $X\in \{1,2,3\}$ and $Y\in \{1,2,3,4\}$.
\item Take $X\sim\Pois{\lambda}$ and $Y\sim\Pois{\mu}$. (Use the chicken-egg story)
\item We can make $X$ and $Y$ such that they are (both) continuous, i.e., have densities.
  The conceptual ideas\footnote{Unless you start digging deeper.
    Then things change drastically, but we skip this technical stuff.}
  don't change much, except that the summations become integrals.
\item Why do people often/sometimes (?)
  model the claim sizes as iid $\sim\Norm{\mu, \sigma^{2}}$?
  There is a slight problem with this model (can real claim sizes be negative?), but what is the way out?
\item The example is more versatile than you might think. Here is another interpretation.

A supermarket has 5 packets of rice on the shelf.
Two customers buy rice, with amounts $X$ and $Y$.
What is the probability of a lost sale, i.e., $\P{X+Y>5}$?
What is the expected amount lost, i.e., $\E{ \max{X+Y - 5,0}}$?

Here is yet another.
Two patients arrive in to the first aid of a hospital.
They need $X$ and $Y$ amounts of service, and there is one doctor.
When both patients arrive at 2 pm, what is the probability that the doctor has work in overtime (after 5 pm), i.e., $\P{X+Y > 5- 2}$?
\end{enumerate}
\end{remark}


\section{Lecture 2}

Read the problems of \verb|memoryless\_excursions.pdf|.
All the problems in that document relate to topics discussed in Sections BH.7.1 and BH.7.2, and quite a lot of topics you have seen in the previous course on probability theory.



\opt{all-solutions-at-end}{
\clearpage
\Closesolutionfile{ans}
\section{Solutions}
\input{ans}
}

\end{document}

%%% Local Variables:
%%% TeX-master: "lectures-demo"
%%% End:
